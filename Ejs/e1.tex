\item Considerar las transformaciones lineales $R_\frac{\pi}{2},S_Y,H_2$ y $P_X$ del ejercicio 11 de la práctica 2. Sea $\E=\{e_1,e_2\}$ la base canónica de $\R^2$.
    \begin{enumerate}
        \item Probar que $Z(e_1,R_\frac{\pi}{2})=\R^2=Z(e_2,R_\frac{\pi}{2})$
            \begin{mdframed}[style=s]
                
            \end{mdframed}
        \item Hallar $Z(e_1,S_Y)$ y $Z(e_2,S_Y)$.
            \begin{mdframed}[style=s]
                
            \end{mdframed}
        \item Probar que $H_2$ no tiene vectores cíclicos.
            \begin{mdframed}[style=s]
                
            \end{mdframed}
        \item Hallar $Z(e_1,P_X)$ y $Z(e_2,P_X)$. ¿Puede dar algún vector cíclico de $P_X$?
            \begin{mdframed}[style=s]
                
            \end{mdframed}
    \end{enumerate}